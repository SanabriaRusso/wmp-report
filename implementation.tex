It is possible to find everything that is needed to run WMP-Editor at the project's Github repository~\cite{WMP-code}. But first, it is worthy to mention the exceptions related to hardware.

\subsection{Hardware compatiblity issues}

As of the time of this report, it is necessary to have a wireless card with the Broadcom chipset $4311$v1 or $4318$. This is not an obstacle for testing WMP-Editor, but in order to test the work it is imperative to have a compliant card. Instructions on how to flash the firmware and other compatibility tasks that must be performed, can be found in the documentation folder of~\cite{WMP-code} (Chapter $8$).

\subsection{Running the WMP-Editor}

WMP-Editor is written in Java, and also can be downloaded from~\cite{WMP-code}. Navigating to \emph{wmp-editor/wmpeditor-v2.37/} reveals the \texttt{WMPEditor.jar} file. In order to execute it, just \texttt{cd} to the mentioned directory and type: \texttt{java -jar WMPEditor.jar} to open WMP-Editor.

Once inside, it is possible to open other \texttt{.sm} files like the ones under \emph{mac-example/dcf/} through the \emph{file} menu on the top left corner. The \texttt{dcf-master.sm} example is shown in Figure~\ref{fig:WMP-EditorLayout}.

\subsection{Current state of our implementation}

Today we had the opportunity of getting a question answered by one of FLAVIA team member. This was related to the adjustment of the backoff value through a \texttt{0x0D TX\_PKT\_SCHEDULER} action parameter, namely \texttt{BACKOFF SLOT}.

As of now, I believe this measure just resets the backoff counter to the specified value. I could not find any other indication that this behavior can be further adjusted. Nevertheless, the option \texttt{STD} of the same \texttt{BACKOFF SLOT} parameter is supposed to mimic the behavior of DCF. 

As for further software requirements: there are a lot of files in the Github repository, and rather fewer are the ones with clear documentation.

In relation to the hardware requirements, the wireless card used in the documentation is called \emph{Broadcom Air Force One}, and what I believe to be a similar model can be found at~\cite{card}. 

Summarizing the questions resulting from this first report:

\begin{enumerate}
	\item \emph{Where is the \texttt{STD} parameter programmed}?
	\item \emph{Can we modify it to behave as our new backoff scheme}?
	\item \emph{Do we have all what we need to try the different MAC protocols provided as examples?}
\end{enumerate}

